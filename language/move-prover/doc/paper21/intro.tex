\Section{Introduction}

The Move Prover (\MVP) is a formal verification tool for smart contracts
that intends to be used routinely during code development.
%
The verification finishes fast and predictably,
making the experience of running \MVP similar to the experience of
running compilers, linters, type checkers, and other development tools.
%
Building a fast verifier is non-trivial, and in this paper,
we would like to share the most important engineering and architectural
decisions that have made this possible.

One factor that makes verification easier is applying it to smart contracts.
%
Smart contracts are easier to verify than conventional software for at least
three reasons:
1) they are small in code size,
2) they execute in a well-defined, isolated environment, and
3) their computations are typically sequential, deterministic, and have
   minimal interactions with the environment
   (e.g., no explicit I/O operations).
%
At the same time,
formal verification is more appealing to the advocates for smart contracts
becaues of the large financial and regulatory risks that smart contracts may
entail if misbehaved,
as evidence by large losses that have occurred already~\cite{???}.

%% Smart contracts have been for longer considered a promising domain to employ
%% formal verification~\cite{CONTRACT_VERIFICATION}.  Smart contracts deal with
%% computations which require highest correctness assurance, as they manipulate
%% assets of large value in an irrevocable way. Those computations need to follow
%% a growing set of rules, reflecting a large diversity of regulations in place,
%% and therefore are vulnerable to bugs.

%% On the other hand, from the viewpoint of verification techniques, smart
%% contracts are relatively \emph{easier} to verify than general programs,
%% because they exist in a well-defined, isolated execution environment.
%% Computations are typically sequential, not i/o-bound, and deterministic.
%% These basic advantages of smart contracts can be further augmented by
%% choice of a verification friendly language and virtual machine design.

The other crucial factor to the success of \MVP is a tight coupling with
the \emph{Move} programming language~\cite{MOVE_LANG}.
%
Move is developed as part of the Diem blockchain~\cite{DIEM} and
is designed to be used with formal verification from day one.
%
Move is currently co-evolving with \MVP.
%
The language supports specifying
pre-, post-, and aborts conditions of functions,
as well as invariants
over data structures and
over the content of the global persistent memory
(i.e., the contents of the blockchain).
%
One feature that makes verification harder is that universal and existential
quantification is used freely in specifications. This makes verification harder,
but reduces errors in specification by allowing users to write properties directly,
without clever encodings that might lead to more errors.

%% In this paper we report about the \emph{Move prover} (abbreviated \MVP), a tool
%% for the \emph{Move} smart contract language~\cite{MOVE_LANG}. Move has been
%% developed together with the Diem blockchain~\cite{DIEM}, and has been designed
%% from the beginning with verification in mind.  The language supports specifying
%% pre, post, and aborts conditions of functions, as well as invariants over data
%% structures and over the content of the global persistent memory (which
%% represents programmable ``storage'' in smart contracts). Specification
%% constructs include universal and existential quantification over arbitrary data
%% types and are therefore generally not decidable.

%% Despite this specification richness,
\MVP is capable of verifying the full
Move implementation of the Diem blockchain
(called the Diem framework~\cite{DIEM_FRAMEWORK})
in a few minutes.
%
The framework provides functionality for managing
accounts and their interactions, including multiple currencies, account roles,
and rules for transactions.
%
It consists of
about 8,800 lines of Move code and
6,500 lines of specifications (including comments for both),
which shows that the framework is extensively specified.
%% approximately 12,000 lines of Move
%% program code and specifications.
More importantly,
\emph{verification is fully automated and runs continuously with unit and integration
  tests},
%% which we consider a significant practical result for formal
%% verification adoption.
which we consider a testament to the practicality of the approach.
%
Running the prover in integration tests requires more than speed:
it requires reliability,
because tests that work sometimes and fail or time out other times
are unacceptable in that context.

\MVP is a substantial and evolving piece of software that has
been tuned and optimized in many ways.
As a result, it is not easy to define exactly what implementation decisions
lead to fast and reliable performance.
%
However, we can at least identify three major ideas that
resulted in dramatic improvements in speed and reliability since the
description of an early prototype of \MVP~\cite{MOVE_PROVER}.
%
The rest of this paper focuses on these three core ideas:
\begin{itemize}
\item an \emph{alias-free memory model} based on Move's semantics, which are similar to the Rust programming language;
\item \emph{fine-grained invariant checking}, which ensures that invariants hold at every state, except when developer explicitly suspends them; and
\item monomorphization, which instantiates type parameters in Move's generic structures, functions, and specification properties.
\end{itemize}
The combined effect of all these improvements transformed a tool that worked, but
often exhibited frustrating, sometimes random~\cite{BUTTERFLY},
timeouts on complex and especially
on erroneous specifications, to a tool that almost always completes in less than 30 seconds.
In addition, there have been many other improvements, including
a more expressive specification language,
reducing false positives,
and error reporting.

%% From the point of the first Move Prover publication in \cite{MOVE_PROVER}, many
%% improvements have been made to make such usage possible.  Those are along speed,
%% predictability, error reporting, and absence of false positives and timeouts.
%% We developed a number of novel translation techniques which optimized SMT
%% performance and, more importantly, resulted in generally more predictable
%% behavior. While in previous versions of the tool, we saw timeouts frequently,
%% plagued by the Butterfly effect~\cite{BUTTERFLY}, the current version only
%% rarely runs into those problems.

The remainder of the paper first introduces the Move language and how \MVP is used with it,
then discusses the design of \MVP and the three main optimizations above.
There is also an
appendix that describes
%% discusses
injection of function specifications, and the mapping to the
Boogie intermediate verification language~\cite{BOOGIE}.

%% This paper is organized as follows. We first give an introduction into the Move
%% language and how \MVP is used with it. We then discuss in more detail the design
%% of \MVP, and the most important translation techniques it uses, including
%% elimination of references from Move programs, evaluating global memory
%% invariants by injection them at the right places into the code, monomorphization
%% of generic programs, and modular verification. For furthergoing study, the
%% appendix discusses injection of function specifications, and the mapping to the
%% Boogie intermediate verification language~\cite{BOOGIE}.


%%% Local Variables:
%%% mode: latex
%%% TeX-master: "main"
%%% End:
