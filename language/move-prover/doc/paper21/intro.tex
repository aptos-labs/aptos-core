\Section{Introduction}

The Move Prover (MVP) is a formal verification tool for smart contracts
that can be used routinely during the development.
It runs about predictably as compilers,
linters, and other development tools.
We would like to share the most important engineering and architectural
decisions that have made this possible.

One factor that made verification easier was applying it to smart contracts.
Additional effort for verification is easy to justify because of the
large financial and regulatory risks that they may entail, as evidence
be large losses that have occurred already~\cite{???}.  Smart
contracts are also easier to verify than conventional software,
because they are often small and because they execute in a
well-defined, isolated environment, and because computations are
typically sequential, not i/o-bound, and deterministic.

* Other keys to success
- coupled with programming language


%% Smart contracts have been for longer considered a promising domain to employ
%% formal verification~\cite{CONTRACT_VERIFICATION}.  Smart contracts deal with
%% computations which require highest correctness assurance, as they manipulate
%% assets of large value in an irrevocable way. Those computations need to follow
%% a growing set of rules, reflecting a large diversity of regulations in place,
%% and therefore are vulnerable to bugs.

%% On the other hand, from the viewpoint of verification techniques, smart
%% contracts are relatively \emph{easier} to verify than general programs,
%% because they exist in a well-defined, isolated execution environment.
%% Computations are typically sequential, not i/o-bound, and deterministic.
%% These basic advantages of smart contracts can be further augmented by
%% choice of a verification friendly language and virtual machine design.

Verification is also easier because we have a new programming language,
\emph{Move}~\cite{MOVE_LANG}, which was developed as part of the
Diem blockchain~\cite{DIEM}, which was designed to be used with formal
verification, and is integrated with MVP.
The language supports specifying
pre-, post-, and aborts conditions of functions, as well as invariants over data
structures and over the content of the global persistent memory
(i.e., the contents of the blockchain).
One feature that makes verification harder is that universal and existential
quantification is used freely in specifications. This makes verification harder,
but reduces errors in specification by allowing users to write properties directly,
without clever encodings that might lead to more errors.

%% In this paper we report about the \emph{Move prover} (abbreviated \MVP), a tool
%% for the \emph{Move} smart contract language~\cite{MOVE_LANG}. Move has been
%% developed together with the Diem blockchain~\cite{DIEM}, and has been designed
%% from the beginning with verification in mind.  The language supports specifying
%% pre, post, and aborts conditions of functions, as well as invariants over data
%% structures and over the content of the global persistent memory (which
%% represents programmable ``storage'' in smart contracts). Specification
%% constructs include universal and existential quantification over arbitrary data
%% types and are therefore generally not decidable.

%% Despite this specification richness,
\MVP is capable of verifying the full
the Move implementation of the Diem
blockchain~\cite{DIEM} (called the Diem framework~\cite{DIEM_FRAMEWORK})
in a few minutes.
The framework provides functionality for managing
accounts and their interaction, including multiple currencies, account roles,
and rules for transactions.  It consists of
about 8,800 lines of Move code and 6,500 lines of specifications (including comments for both).
%% approximately 12,000 lines of Move
%% program code and specifications.
The framework is %% exhaustively
extensively specified.
Importantly,
\emph{verification is fully automatic and is run interactively with unit and integration
  tests},
%% which we consider a significant practical result for formal
%% verification adoption.
which we consider a testament to the practicality of the approach.
Running the prover in integration tests requires more than speed: It requires
reliability, because tests that work sometimes and fail or time out other
times are unacceptable in that context.

The Prover is a substantial and evolving piece of software that has
been tuned and optimized in many ways, so it is not easy to define
exactly what implementation decisions lead to fast and reliable
performance.  However, we can identify three major ideas that
resulted in dramatic improvements in speed and reliability since the
description of an early prototype of \MVP~\cite{MOVE_PROVER},
which are the focus of the rest of this paper:
\begin{itemize}
\item An \emph{alias-free memory model}, based on Move's semantics, which are similar to the Rust programming language;
\item \emph{fine-grained invariant checking}, which ensures that invariants hold at every state, except when developer explicitly suspends them;
\item and monomorphization, which instantiates type parameters in Move's generic
  structures and functions
\end{itemize}
The combined effect of all these improvements transformed a tool that worked, but
often exhibited frustrating, sometimes random~\cite{BUTTERFLY},
timeouts on complex and especially
on erroneous specifications, to a tool that almost always completes in less than 30 seconds.
In addition, there have been many other improvements, including in the specification language,
reducing false positives, and error reporting.

%% From the point of the first Move Prover publication in \cite{MOVE_PROVER}, many
%% improvements have been made to make such usage possible.  Those are along speed,
%% predictability, error reporting, and absence of false positives and timeouts.
%% We developed a number of novel translation techniques which optimized SMT
%% performance and, more importantly, resulted in generally more predictable
%% behavior. While in previous versions of the tool, we saw timeouts frequently,
%% plagued by the Butterfly effect~\cite{BUTTERFLY}, the current version only
%% rarely runs into those problems.

The remainder of the paper first introduces the Move language and how \MVP is used with it,
then discusses the design of \MVP and the three main optimizations above.
There is also an
appendix that describes
%% discusses
injection of function specifications, and the mapping to the
Boogie intermediate verification language~\cite{BOOGIE}.

%% This paper is organized as follows. We first give an introduction into the Move
%% language and how \MVP is used with it. We then discuss in more detail the design
%% of \MVP, and the most important translation techniques it uses, including
%% elimination of references from Move programs, evaluating global memory
%% invariants by injection them at the right places into the code, monomorphization
%% of generic programs, and modular verification. For furthergoing study, the
%% appendix discusses injection of function specifications, and the mapping to the
%% Boogie intermediate verification language~\cite{BOOGIE}.


%%% Local Variables:
%%% mode: latex
%%% TeX-master: "main"
%%% End:
