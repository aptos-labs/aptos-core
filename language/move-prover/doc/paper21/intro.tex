\Section{Introduction}

Smart contracts have been for long considered a promising domain to employ
formal verification.  On the one hand, smart contracts deal with computations
which require highest correctness assurance, as they may manipulate assets of
large value with many rules and regulations in place. On the other hand, smart
contracts have multiple properties (sandboxed, transactional, bounded
computation steps) which make them technically well-suited for formal
verification.

In this paper we report about the \emph{Move prover} (abbreivated \MVP), a tool
for the \emph{Move} smart contract language~\cite{MOVE_LANG}, which has been
developed together with the Diem blockchain~\cite{DIEM}. The Move language
supports specifying pre- and post-conditions of functions, as well as invariants
over data structures and over the content of the global persistent memory (which
represents programmable ``storage'' in smart contracts). Specification
constructs include universal and existential quantification over arbitrary data
types and are therefore generally not decidable.

Despite these theoretical restrictions, \MVP is capable of verifying the full
Diem framework~\cite{DIEM_FRAMEWORK}, which is the Move implementation of the
Diem blockchain~\cite{DIEM}. The framework provides functionality for managing
accounts and their interaction, including multiple currencies, account roles,
and rules for transactions.  It consists of approximately 12,000 lines of Move
program code and specifications.  The framework is exhaustively specified, and
\emph{verification runs fully automated alongside with unit and integration
  tests, typically in seconds}, which can be seen as a significant practical
result for formal verification adoption.

From the point of the first Move Prover publication in \cite{MOVE_PROVER}, many
improvements have been made to make such usage practical.  Those are along
speed, predictability, error reporting, and absence of false positives and
timeouts.  We developed a number of novel translation techniques which optimized
SMT performance by an order of magnitude, and more importantly, resulted in
generally more predictable behavior. While in previous versions of the tool, we
saw timeouts frequently, plagued by the Butterfly effect~\cite{BUTTERFLY}, the
current version only rarely runs in those problems.

This paper is organized as follows. We first give an introduction into the Move
language and how \MVP is used with it. We then discuss in more detail the design
of \MVP, and the most important translation techniques it uses, including
elimination of references from Move programs, evaluating global memory
invariants by injection them at the right places into the code, monomorphization
of generic programs, and modular verification.




%%% Local Variables:
%%% mode: latex
%%% TeX-master: "main"
%%% End:
