\Section{Introduction}

Smart contracts have been for longer considered a promising domain to employ
formal verification~\cite{CONTRACT_VERIFICATION}.  Smart contracts deal with
computations which require highest correctness assurance, as they manipulate
assets of large value in an irrevocable way. Those computations need to follow
a growing set of rules, reflecting a large diversity of regulations in place,
and therefore are vulnerable to bugs.

On the other hand, from the viewpoint of verification techniques, smart
contracts are relatively \emph{easier} to verify than general programs,
because they exist in a well-defined, isolated execution environment.
Computations are typically sequential, not i/o-bound, and deterministic.
These basic advantages of smart contracts can be further augmented by
choice of a verification friendly language and virtual machine design.

In this paper we report about the \emph{Move prover} (abbreviated \MVP), a tool
for the \emph{Move} smart contract language~\cite{MOVE_LANG}. Move has been
developed together with the Diem blockchain~\cite{DIEM}, and has been designed
from the beginning with verification in mind.  The language supports specifying
pre, post, and aborts conditions of functions, as well as invariants over data
structures and over the content of the global persistent memory (which
represents programmable ``storage'' in smart contracts). Specification
constructs include universal and existential quantification over arbitrary data
types and are therefore generally not decidable.

Despite this specification richness, \MVP is capable of verifying the full Diem
framework~\cite{DIEM_FRAMEWORK}, the Move implementation of the Diem
blockchain~\cite{DIEM}. The framework provides functionality for managing
accounts and their interaction, including multiple currencies, account roles,
and rules for transactions.  It consists of approximately 12,000 lines of Move
program code and specifications.  The framework is exhaustively specified, and
\emph{verification runs fully automated alongside with unit and integration
  tests}, which we consider a significant practical result for formal
verification adoption.

From the point of the first Move Prover publication in \cite{MOVE_PROVER}, many
improvements have been made to make such usage possible.  Those are along speed,
predictability, error reporting, and absence of false positives and timeouts.
We developed a number of novel translation techniques which optimized SMT
performance and, more importantly, resulted in generally more predictable
behavior. While in previous versions of the tool, we saw timeouts frequently,
plagued by the Butterfly effect~\cite{BUTTERFLY}, the current version only
rarely runs into those problems.

This paper is organized as follows. We first give an introduction into the Move
language and how \MVP is used with it. We then discuss in more detail the design
of \MVP, and the most important translation techniques it uses, including
elimination of references from Move programs, evaluating global memory
invariants by injection them at the right places into the code, monomorphization
of generic programs, and modular verification. For furthergoing study, the
appendix discusses injection of function specifications, and the mapping to the
Boogie intermediate verification language~\cite{BOOGIE}.


%%% Local Variables:
%%% mode: latex
%%% TeX-master: "main"
%%% End:
