\SubSection{Specification Injection}

Move specifications are reduced to basic assume/assert statements injected into
the Move code. As usual, an assume statement formulates a condition which can be
assumed to hold for verification at a given program point, whereas an assert
statement a condition which needs to be verified.


Specification instrumentation starts from a program for which references have
been removed, and mutation is transformed into a read-update-write cycle
(Sec.~\ref{sec:RefElim}).


\SubSubSection{Modular Verification}
\label{sec:ModularVerification}

Modular verification applies to all types of injections, and its principles are
therefore described first. When the Move prover is run, it takes as input a set
of Move modules which is closed under the transitive dependency relation (module
imports). However, only a subset of those modules are \emph{verification target}
(typically just one module). It is assumed that the tool environment ensures
that modules in the dependency relation which are not target of verification
have already successfully verified.

From the set of target modules, the set of \emph{target functions} is
derived. This set might be enriched by additional functions which need
verification because of global invariants, as discussed in
Sec.~\ref{sec:GlobalInvariants}. The resulting set of target functions will then
be verified one-by-one, assuming that any called functions have successfully
verified. If a called function is among the target functions, it might in fact
not verify; however, in this case a verification error will be reported at the
called function, and the verification result at the caller side can be ignored.

\SubSubSection{Function Specifications}

The injection of basic function specifications is illustrated in
Fig.~\ref{fig:RequiresEnsuresAbortsIf}.  An extension of the Move source
language is used to specify abort behavior. With~%
|fun f() { .. } onabort { conditions }| a Move function is defined where
|conditions| are assume or assert statements that are evaluated at every program
point the function aborts (either implicitly or with an |abort| statement). This
construct simplifies the presentation and corresponds to a per-function abort
block on bytecode level which is target of branching.

\begin{figure}[t!]
  \caption{Requires, Ensures, and AbortsIf Injection}
  \label{fig:RequiresEnsuresAbortsIf}
  \centering
\begin{MoveBox}
  fun f(x: u64, y: u64): u64 { x + y }
  spec f {
    requires x < y;
    aborts_if x + y > MAX_U64;
    ensures result == x + y;
  }
  fun g(x: u64): u64 { f(x, x + 1) }
  spec g {
    ensures result > x;
  }
  @\transform@
  fun f(x: u64, y: u64): u64 {
    spec assume x < y;
    let result = x + y;
    spec assert result == x + y;     // ensures of of
    spec assert                      // negated abort_if of f
      !(x + y > MAX_U64); @\label{line:aborts_holds_not}@
    result
  } onabort {
    spec assert                      // abort_if of f
      x + y > MAX_U64; @\label{line:aborts_holds}@
  }
  fun g(x: u64): u64 {
    spec assert x < x + 1;           // requires of f
@$\textrm{\it if inlined}$\label{line:inline}@
    let result = inline f(x, x + 1);
@$\textrm{\it elif opaque}$\label{line:opaque}@
    if (x + x + 1 > MAX_U64) abort;  // aborts_if of f
    spec assume result == x + x + 1; // ensures of f
@$\textrm{\it endif}$@
    spec assert result > x;          // ensures of g
    result
  }
\end{MoveBox}
\end{figure}

An aborts condition is translated into two different asserts: one where the
function aborts and the condition must hold (line~\ref{line:aborts_holds}), and
one where it returns and the condition must \emph{not} hold
(line~\ref{line:aborts_holds_not}). If there are multiple |aborts_if|, they are
or-ed. If there is no aborts condition, no asserts are generated. This means
that once a user specifies aborts conditions, they must completely cover the
abort behavior of the code. (The prover also provides an option to relax this
behavior, where aborts conditions can be partial and are only enforced on
function return.)

For a function call site we distinguish two variants: the call is \emph{inlined}
(line~\ref{line:inline}) or it is \emph{opaque} (line~\ref{line:opaque}). In
both cases, it is assumed that the called function is verified (see Modular
Verification, Sec.~\ref{sec:ModularVerification}). For inlined calls, the
function definition, with all injected assumptions and assertions turned into
assumptions (as those are considered proven) is substituted. For opaque
functions the specification conditions are inserted as
assumptions. Methodlogically, opaque functions need precise specificatons
relative to a particular objective, where as in the case of inlined functions
the code is still the source of truth and specifications can be partial or
omitted. However, inlining does not scale arbitrarily, and can be only used for
small function systems.

\Paragraph{Modifies Condition}

\begin{figure}[t!]
  \caption{Modifies Injection}
  \label{fig:Modifies}
  \centering
\begin{MoveBox}
  fun f(addr: address) { move_to<T>(addr, T{}) }
  spec f {
    pragma opaque;
    ensures exists<T>(addr);
    modifies global<T>(addr);
  }
  fun g() { f(0x1) }
  spec g {
    modifies global<T>(0x1), global<T>(0x2);
  }
  @\transform@
  fun f(addr: address) {
    let can_modify_T = {addr};         // modifies of f
    spec assert addr in can_modify;    // permission check move_to @%
                                            \label{line:modifies_permission}@
    move_to<T>(addr, T{});
  }
  fun g() {
    let can_modify_T = {0x1, 0x2};     // modifies of g
    spec assert {0x1} <= can_modify_T; // permission check call f @%
                                            \label{line:modifies_call_permission}@
    spec havoc global<T>(0x1);         // havoc memory modified by f @%
                                            \label{line:modifies_havoc}@
    spec assume exists<T>(0x1);        // ensures of f
  }
\end{MoveBox}
\end{figure}


The |modifies| condition, part of a function specification, is implemented in a
particular way.  Part of it is realized by a static type check analysis, and
part of it injected as assertions. Fig.~\ref{fig:Modifies} illustrates the
approach.

The type check ensures that if a function specificies a~%
|modifies global<T>(addr)|, then all called functions which are \emph{opaque}
declare modifies for the same type |T|. This is important so we can relate the
callees memory modifications to that what is allowed at caller side.

At verification time, when an operation is performed which modifies memory, an
assertion is emitted that modification is allowed
(e.g. line~\ref{line:modifies_permission}). The permitted addresses derived from
the modifies clause are stored in a set |can_modify_T| generated by the
transformation. Instructions which modify memory are either primitives (like
|move_to| in the example) or function calls. If the function call is inlined,
modifies injection proceeds (conceptually) with the inlined body. For opaque
function calls, the static analysis has ensured that the target has a modifies
clause.  This clause is used to derive the modified memory, which must be a
subset of the modified memory of the caller
(line~\ref{line:modifies_call_permission}).

For opaque calls, we also need to \emph{havoc} the memory they modify
(line~\ref{line:modifies_havoc}), by which is meant assigning an unconstraint
value to it. If present, |ensures| from the called function, injected as
subsequent assumptions, are then constraining it.


\SubSubSection{Data Invariants}

\begin{figure}[t!]
  \caption{Data Invariant Injection}
  \label{fig:DataInvariants}
  \centering
\begin{MoveBox}
  struct S { a: u64, b: u64 }
  spec S { invariant a < b }
  fun f(s: S): S { let r = &mut s; r.a = r.a + 1; r.b = r.b + 1; s }
  @\transform@
  fun f(s: S): S {
    spec assume s.a < s.b;      // assume invariant for parameter
    let r = Mut::local(s, F_s); // begin mutation of s
    r = Mut::set(r, Mut::get(r)[a = Mut::get(r).a + 1]);
    r = Mut::set(r, Mut::get(r)[b = Mut::get(r).b + 1]);
    spec assert                 // end mutation: invariant enforced
      Mut::get(r).a < Mut::get(r).b;
    s = Mut::get(r);            // write back to s
    s
  }
\end{MoveBox}
\end{figure}

A data invariant specifies a constraint over a struct value. The value is
guaranteed to satisfy this constraint at any time. Thus, when a value is
constructed, the data invariant needs to be verified, and when it is consumed,
it can be assumed to hold.

In Move's reference semantics, construction of struct values is often done via a
sequence of mutations via mutable references. It is desirable that \emph{during}
such mutations, assertion of the data invariant is suspended. This allows to
state invariants which reference multiple fields, where the fields are updated
step-by-step.  Move's borrow semantics and concept of mutations provides a
natural way how to defer invariant evaluation: at the point a mutable reference
is released, mutation ends, and the data invariant can be enforced.  In other
specification formalisms, we would need a special language construct for
invariant suspension. Fig.~\ref{fig:DataInvariants} gives an example, and shows
how data invariants are reduced to assert/assume statements.

\Paragraph{Implementation}

The implementation hooks into the reference elimination
(Sec.~\ref{sec:RefElim}). As part of this the lifetime of references is
computed. Whenever a reference is released and the mutated value is written
back, we also enforce the data invariant. In addition, the data invariant is
enforced when a struct value is directly constructed.







\SubSubSection{Global Invariants}
\label{sec:GlobalInvariants}




%%% Local Variables:
%%% mode: latex
%%% TeX-master: "main"
%%% End:
